\documentclass[
 %reprint,
 aps, pra,
 amsmath,amssymb,
 11pt,
 final,
% notitlepage,
tightenlines,
 %twoside,
 %twocolumn,
 nofloats,
% nobibnotes,
nofootinbib,
 superscriptaddress,
%noshowpacs,
showkeys,
showkeywords,
%centertags
 ]
{revtex4-2}

\usepackage[T2A]{fontenc}
\usepackage[utf8x]{inputenc}
\usepackage[russian,english]{babel}
\usepackage{graphicx}% Include figure files
\usepackage{dcolumn}% Align table columns on decimal point
\usepackage{bm}% bold math
%\usepackage{longtable}


\input{maik.rty}


%\renewcommand{\rmdefault}{lh}

\setcitestyle{authoryear,round}
\setlength{\bibhang}{1.5em}
% Начало документа
\begin{document}

\selectlanguage{russian}

\keywords{звезды: двойные и кратные---звезды: индивидуальные: ADS\,48}

\title{НАЗВАНИЕ СТАТЬИ ЗАГЛАВНЫМИ БУКВАМИ}

\author{\firstname{И.~О.}~\surname{Автор1}}
 \email{example@list.ru}
 \affiliation{Главная (Пулковская) астрономическая обсерватория
РАН, Санкт-Петербург, 196140 Россия}

\author{\firstname{И.~О.}~\surname{Автор2}}
 \affiliation{Казанский (Приволжский) федеральный университет, Казань, 420008 Россия}

\begin{abstract}
	В данной статье рассматривается анализ чувствительности плоскопараллельных манипуляторов к изменениям их геометрических параметров, вызванных ошибками при монтаже манипулятора. На примере трехточечного плоско-паралельного манипулятора 3-RPR...
\end{abstract}

\maketitle


\section{ВВЕДЕНИЕ}


\section{ЗАКЛЮЧЕНИЕ}

В данной работе демонстрируется возможность\ldots

\selectlanguage{russian}

\include{chapters/intro.tex}
\newpage

\section{Архитектура робота}

Манипуляторы с параллельной кинематикой представляют собой роботизированные устройства с улучшенными кинематическими и динамическими характеристиками, в следствии чего находят всё большее применение на производственных линиях.

\begin{acknowledgments}
	Авторы признательны\ldots
\end{acknowledgments}

%\bibliographystyle{aspb1}
%\bibliography{Kiyaeva}
\include{chapters/bibliography.tex}

\end{document}